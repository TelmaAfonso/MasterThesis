
\chapter*{Agradecimentos}

Este espaço é dedicado a todos aqueles que deram a sua valiosa contribuição para que esta dissertação pudesse ser concretizada. A todos eles o meu mais sincero agradecimento.

Em especial, ao professor Miguel Rocha pela orientação, por toda a partilha de conhecimento e todos os valiosos conselhos, não só no decorrer deste trabalho, mas durante os dois anos do mestrado. Agradeço também todas as oportunidades que me proporcionou durante este período, que contribuíram para o meu crescimento não só em termos académicos, mas também como pessoa. Agradeço, sobretudo, o voto de confiança.

Agradeço também ao professor Marcelo Maraschin, pela sua orientação e valiosas sugestões, que contribuíram para o sucesso deste trabalho. Agradeço ainda a oportunidade de poder ter trabalhado em parceria com a Universidade Federal de Santa Catarina, da qual resultou a publicação de um artigo que constitui agora um capítulo nesta dissertação. Artigo este que foi possível em grande parte devido ao Rodolfo, com toda a sua ajuda e disponibilidade e, por isso, a ele deixo aqui também o meu agradecimento. 

Ao Helder, um obrigada especial por todas as palavras de apoio, pelo carinho, por ter alegrado os meus dias durante este percurso, tendo sempre as palavras certas na hora de aconselhar e criticar.

A todos os meus amigos, um enorme obrigada pelos tão necessários momentos de descontração, alegria e sobretudo de diversão.

Aos membros do grupo \gls{bisbii} aqui da Universidade do Minho, agradeço o facto de estarem sempre prontos a ajudar e por tornarem os dias mais divertidos, contribuindo assim para um bom ambiente de trabalho.

Por último, deixo aqui o meu enorme agradecimento a toda a minha família, aos meus pais e à minha irmã, por me terem incentivado durante todos estes anos e por terem possibilitado que eu hoje chegasse aqui. Sem eles não seria o que sou hoje. 

A todos vós, dedico este trabalho.
