
\section{Available Free Tools For Metabolomics and Spectral Data} \label{met_tools}

In response to the increasing growth of information output through the years, a number of computational tools for metabolomics and spectral data have become available. Among these are some interesting packages on the open-source R scientific computing platform (\href{http://www.r-project.org}{\nolinkurl{http://www.r-project.org}}), including \textit{hyperSpec} and \textit{ChemoSpec}. 

The \textit{hyperSpec} package allows convenient handling of Hyperspectral data (i.e. spectra with associated space, time or other additional information). It handles data recorded over a discretized axis, obtained from \gls{uv}, \gls{nir}, \gls{ir}, Raman and other spectroscopy techniques. It has several plot functions to display spectra, false-colour maps, calibration curves, among other purposes. Preprocessing methods include data normalization, intensity calibration, offset and baseline corrections, spectral interpolation, and many other methods. For the analysis, this package provides functions for clustering analysis, \gls{pca}, \gls{pls}, \gls{lda}, among others \citep{hyperspec}.

The \textit{ChemoSpec} package was designed with metabolomics data sets in mind, where the samples fall into groups, such as treatment and control. It consists in a collection of functions for entirely exploratory and unsupervised data analysis of spectral data, including \gls{ir}, \gls{uv} and Raman data, among other types of spectral data. It includes functions for plotting and inspecting spectra, as well as some pre-processing functions for data normalization and binning, identifying and removing problematic samples or regions of no interest, baseline correction, peak alignment, among others. Unsupervised methods such as \gls{hca}, \gls{pca} and model-based clustering are also covered in this package \citep{chemospec}.

Among web-based tools available for metabolomics and spectral data analysis the most notable and comprehensive is \textit{MetaboAnalyst}. It accepts data from either targeted profiling (concentration tables) or metabolic fingerprinting approaches (spectral bins, peak lists) produced from either \gls{nmr}, \gls{lcms} or \gls{gcms}. Preprocessing of the data is available, including normalization and scaling, data transformation, outlier removal, among other methods. The statistical analysis module offers various commonly used statistical and machine learning methods, including t-tests, \gls{anova}, \gls{pca}, \gls{plsda}, Orthogonal \gls{plsda}, and also clustering and visualization tools to create dendrograms and heatmaps as well as to classify based on random forests and \gls{svm}. \textit{MetaboAnalyst} includes modules for other types of analysis as well, including the enrichment analysis, pathway and time-series analysis modules, among others. These and other freely available tools are listed in \autoref{tools}. 

\begin{table}[h]
	\caption{Available free tools for metabolomics and spectral data.} 
	\label{tools}
	\begin{scriptsize}
		\tabulinesep=1.2mm
		\begin{tabu}{|m{1.8cm}|m{5.2cm}|m{5cm}|m{2.4cm}|}		
			\rowcolor{airforceblue}
			\htab{Name} & \htab{URL} & \htab{Description} & \htab{Data types} \\
			\hline
			
			chemometrics & \href{https://CRAN.R-project.org/package=chemometrics}{\nolinkurl{https://CRAN.R-project.org/package=chemometrics}} & R package for multivariate statistical analysis in chemometrics & Chemical data \\ 				
			\hline 
			
			ChemoSpec & \href{https://CRAN.R-project.org/package=ChemoSpec}{\nolinkurl{https://CRAN.R-project.org/package=ChemoSpec}} & R package for exploratory analysis of spectral data & \gls{nmr}, \gls{ir} and Raman  \\ 		
			\hline 
			
			COLMAR & \href{https://spin.ccic.ohio-state.edu/index.php/colmar}{\nolinkurl{https://spin.ccic.ohio-state.edu/index.php/colmar}} & Webserver for \gls{nmr} data analysis and compound identification & \gls{nmr} \\ 
			
			\hline 
			hyperSpec & \href{https://CRAN.R-project.org/package=hyperSpec}{\nolinkurl{https://CRAN.R-project.org/package=hyperSpec}} & R package for Hyperspectral data analysis & \gls{uv}, \gls{ir}, \gls{nmr}, \gls{ms}, Raman, ... \\ 
			
			\hline 
			MeltDB & \href{https://meltdb.cebitec.uni-bielefeld.de/cgi-bin/login.cgi}{\nolinkurl{https://meltdb.cebitec.uni-bielefeld.de/cgi-bin/login.cgi}} & Web-based system for metabolomics data analysis and dataset annotation & \gls{gcms} and \gls{lcms} \\ 
			
			\hline 
			MetaboAnalyst & \href{http://www.metaboanalyst.ca}{\nolinkurl{http://www.metaboanalyst.ca}} & Web-based system for metabolomics and spectral data analysis & \gls{nmr}, \gls{lcms} and \gls{gcms} \\ 
			
			\hline
			metabolomics & \href{https://CRAN.R-project.org/package=metabolomics}{\nolinkurl{https://CRAN.R-project.org/package=metabolomics}} & R package for metabolomics data analysis & \gls{nmr}, \gls{gcms}, \gls{lcms} and \gls{ms} \\
			
			\hline 
			MetaboMiner & \href{https://wishart.biology.ualberta.ca/metabominer/}{\nolinkurl{https://wishart.biology.ualberta.ca/metabominer/}} & Java based software for \gls{nmr} data analysis and compound identification & \gls{nmr} \\ 
			
			\hline 
			metaP-Server & \href{http://metap.helmholtz-muenchen.de/metap2/}{\nolinkurl{http://metap.helmholtz-muenchen.de/metap2/}} & Web-based system for metabolomics data analysis & \gls{nmr}, \gls{gcms}, \gls{lcms} and \gls{ms} \\ 
			
			\hline 
			muma & \href{https://CRAN.R-project.org/package=muma}{\nolinkurl{https://CRAN.R-project.org/package=muma}} & R package for metabolomics univariate and multivariate analysis & \gls{nmr}, \gls{gcms}, \gls{lcms} and \gls{ms} \\ 
			
			\hline 
			MVAPACK & \href{https://bionmr.unl.edu/mvapack.php}{\nolinkurl{https://bionmr.unl.edu/mvapack.php}} & Toolkit for \gls{nmr} and \gls{ms} data handling & \gls{nmr} and \gls{ms} \\ 
			
			\hline 
			OpenMS & \href{https://www.openms.de/}{\nolinkurl{https://www.openms.de/}} & C++ library for \gls{lcms} data handling and analysis & \gls{lcms} \\ 
			
			\hline 
			specmine & \href{https://CRAN.R-project.org/package=specmine}{\nolinkurl{https://CRAN.R-project.org/package=specmine}} & R package for the integrated analysis of metabolomics and spectral data & \gls{nmr}, \gls{gcms}, \gls{lcms}, \gls{uv}, \gls{ir}, Raman, ... \\ 
			
			\hline 		
		\end{tabu}
	\end{scriptsize}
\end{table}

Of all the listed tools, only some of the R packages include functions for spectral data such as \gls{ir}, \gls{uv} and Raman data, while none of the web-based tools address these types of data. This is a major disadvantage among these web-based tools, since they mostly work with \gls{nmr} and chromatography data, leaving the already mentioned spectroscopies techniques aside, which are commonly employed in metabolic fingerprinting experiments. The \textit{specmine} package provides a set of methods for metabolomics data analysis, including data loading in different formats, pre-processing, metabolite identification, univariate and multivariate data analysis, machine learning and feature selection. It was the base for the development of this work and, therefore, will be discussed in detail in \autoref{specmine_chapter}.



\section{Other General Free Tools}

There are other freely available tools that, although not specific for metabolomics analysis tasks, may be of great value for such analysis. These include, for instance, packages on the open-source R scientific computing platform (\href{http://www.r-project.org}{\nolinkurl{http://www.r-project.org}}), namely \textit{rminer} and \textit{caret} packages.

The \textit{rminer} package facilitates the use of data mining algorithms in classification and regression tasks by presenting a short and coherent set of functions. \textit{rminer} currently has 16 classification and 18 regression methods available, including functions for \gls{plsr}, \gls{lda}, random forests, \gls{svm} and \gls{knn} methods, among others. \gls{tsf} is also included, which is a special case of regression and involves the analysis of a time ordered phenomenon. This package offers a large range of evaluation metrics and graphs that can be used to evaluate the quality of the fitted models and extract knowledge learned from the data-driven models. It also adjusts the hyperparameters of the models, performing some feature selection methods \citep{rminer}. 

The \textit{caret} package consists in set of functions for training and plotting classification and regression models. It contains some functions for data visualization in the form of boxplots, scatter and density plots, as well as data pre-processing functions for data centering and scaling, missing values handling and data transformation, among others. \textit{Caret} also has feature selection functions, implementing both filter and wrapper approaches. Available regression and classification methods in this package include \gls{plsr}, \gls{lda}, \gls{knn}, \gls{svm}, neural networks, random forests, among many others. Tuning model parameters is also possible, with functions that allow to choose the best set of parameters for a given model, estimating the model performance using a training set \citep{caret}.



% ML Bench???

%\hline 
%rminer & \href{https://CRAN.R-project.org/package=rminer}{\nolinkurl{https://CRAN.R-project.org/package=rminer}} & R package with functions for data mining classification and regression methods \\ 

%caret
%https://CRAN.R-project.org/package=caret
%Misc functions for training and plotting classification and regression models.

