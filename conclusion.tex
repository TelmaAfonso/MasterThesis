
\chapter{Conclusions and Future Work}

A number of tools for metabolomics and spectral data analysis have been put forward recently, being one of the major limitations still faced the lack of integrated frameworks for extraction of relevant knowledge and the the lack of reproducibility in many data analysis. The \textit{specmine} R package addresses some of these issues, but it could prove difficult to use for those without programming knowledge. 

In this work, a web platform for spectral data analysis and mining based on the \textit{specmine} package was developed, providing an easier and more user friendly interface, while also addressing some of the package's current limitations.

The developed web platform includes modules for a variety of important
aspects of spectral data analysis, starting with data loading to create useful datasets for further manipulation. Modules for data pre-processing and visual exploration are also available, while also including a data analysis module which covers a variety of methods from univariate analysis to, for instance, regression analysis. The different modules were validated using real data from the literature, attesting the platforms robustness and utility.

One important feature present in the platform is the authentication system that gives access to the user's personal projects library, where a variety of projects can be stored for easy and quick analysis pipeline creation. The workspace containing all created datasets and analysis can also be saved/loaded at any given time, provided that the user is authenticated.

With this being said I believe that the developed web platform will be of valuable use for researchers who wish to perform easy, reliable and reproducible analysis pipelines, on a scientific level.

Improvements could, however, be made to the platform, including the implementation of every feature available in \textit{specmine}, giving the platform the same level of data manipulation as the package; making figures/plots more customizable; adding more analysis types; and make it more robust overall.


