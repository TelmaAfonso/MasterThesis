
\chapter{Conclusions and Future Work}

A number of tools for metabolomics and spectral data analysis have been put forward recently, being one of the major limitations still faced the lack of integrated frameworks for extraction of relevant knowledge and the the lack of reproducibility in many data analysis. The \textit{specmine} R package addresses some of these issues, but it could prove difficult to use for those without programming knowledge. 

In this work, a web platform for spectral data analysis and mining based on the \textit{specmine} package was developed, providing an easier and more user friendly interface, while also addressing some of the package's current limitations.

The developed web platform includes modules for a variety of important
aspects of spectral data analysis, starting with data loading to create useful datasets for further manipulation. Modules for data pre-processing and visual exploration are also available, while also including a data analysis module which covers a variety of methods from univariate analysis to, for instance, regression analysis. The different modules were validated using real data from previously published studies in the host group, attesting the platform's robustness and utility.

One important feature present in the platform is the authentication system that gives access to the user's personal projects library, where a variety of projects can be stored for easy and quick analysis pipeline creation. The workspace containing all created datasets and analyses can also be saved/loaded at any given time, provided that the user is authenticated.

With this being said, the developed web platform meets all conditions to be of valuable use for researchers who wish to perform easy, reliable and reproducible analysis pipelines, on a scientific level.

There are still, however, many aspects that could be improved in the web platform, including the improvement of already existing features and the implementation of new ones. Future work could include:


\begin{itemize}
	\item Implementation of every feature available in \textit{specmine}, giving the platform the same level of data manipulation as the package;
	\item Make figures/plots more customizable;
	\item Add new types of analyses;
	\item Improve the graphical interface of \textit{My Projects} and \textit{Public Projects} pages;
	\item Implement an in-app messaging system to allow the communication between users and results discussion;
	\item Implement a bug reporting feature;
	\item Optimize code execution to reduce page navigation times (probably an  underlying problem of \textit{shiny} for extensive applications);
	\item Improve the overall robustness of the web platform (e.g. improve error catching for all methods);
	\item Add methods to more automatically load datasets from existing databases and other resources.
\end{itemize}

Of course, there are also a large number of features that may be implemented in the core \textit{specmine} package and, then, added to the web user interface.
Although a full list is out of scope of this thesis, areas of future developments are pathway analysis, biomarker identification and integration with other types of omics data.


